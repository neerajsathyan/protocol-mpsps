% This is samplepaper.tex, a sample chapter demonstrating the
% LLNCS macro package for Springer Computer Science proceedings;
% Version 2.20 of 2017/10/04
%
\documentclass[runningheads]{llncs}
%
\usepackage{graphicx}
\usepackage{hyperref}
\usepackage{listings}
% Used for displaying a sample figure. If possible, figure files should
% be included in EPS format.
%
% If you use the hyperref package, please uncomment the following line
% to display URLs in blue roman font according to Springer's eBook style:
% \renewcommand\UrlFont{\color{blue}\rmfamily}
% \hypersetup{
%     colorlinks=true,
%     linkcolor=blue,
%     filecolor=magenta,      
%     urlcolor=cyan,
% }

\begin{document}
%
\title{Protocol Validation Homework Assignment}
\subtitle{Movable Patient Support for an MRI Scanner}
%
%\titlerunning{Abbreviated paper title}
% If the paper title is too long for the running head, you can set
% an abbreviated paper title here
%
\author{Mullai\inst{1}\orcidID{2701556} \and
Neeraj Sathyan\inst{1}\orcidID{2660192}}
%
\authorrunning{Group 71}
% First names are abbreviated in the running head.
% If there are more than two authors, 'et al.' is used.
%
\institute{Vrije Universiteit Amsterdam
\url{https://www.cs.vu.nl/}}
%
\maketitle              % typeset the header of the contribution
%
% \begin{abstract}
% The evaluation and analysis report of assignment 1 of the course 'Data Mining Techniques'.

% \end{abstract}


\label{Introduction}

\label{task1}
\section{Introduction}

\break
\label{task2}
\section{Global Requirements}
\break
\label{task3}
% 3. Translate the global requirements in terms of these interactions.
\section{Interactions of Control Systems}

\break
\label{task4}
\section{Translation of global requirements in terms of the requirements}
\break

\end{document}
