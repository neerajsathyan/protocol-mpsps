% This is samplepaper.tex, a sample chapter demonstrating the
% LLNCS macro package for Springer Computer Science proceedings;
% Version 2.20 of 2017/10/04
%
\documentclass[runningheads]{llncs}
%
\usepackage{graphicx}
\usepackage{hyperref}
\usepackage{listings}
% Used for displaying a sample figure. If possible, figure files should
% be included in EPS format.
%
% If you use the hyperref package, please uncomment the following line
% to display URLs in blue roman font according to Springer's eBook style:
% \renewcommand\UrlFont{\color{blue}\rmfamily}
% \hypersetup{
%     colorlinks=true,
%     linkcolor=blue,
%     filecolor=magenta,      
%     urlcolor=cyan,
% }

\begin{document}
%
\title{Protocol Validation Homework Assignment}
\subtitle{Movable Patient Support for an MRI Scanner}
%
%\titlerunning{Abbreviated paper title}
% If the paper title is too long for the running head, you can set
% an abbreviated paper title here
%
\author{Mullai\inst{1}\orcidID{2701556} \and
Neeraj Sathyan\inst{1}\orcidID{2660192}}
%
% First names are abbreviated in the running head.
% If there are more than two authors, 'et al.' is used.
%
\institute{Vrije Universiteit Amsterdam
\url{https://www.cs.vu.nl/}}
%
\maketitle              % typeset the header of the contribution
%
% \begin{abstract}
% The evaluation and analysis report of assignment 1 of the course 'Data Mining Techniques'.

% \end{abstract}


\label{Introduction}

\label{task1}
\section{Introduction}

\break
\label{task2}
\section{Global Requirements}
\break
\label{task3}
% 3. Translate the global requirements in terms of these interactions.
\section{Interactions of Control Systems}

\break
\label{task4}
\section{Translation of global requirements in terms of the requirements}
\break
\label{reflections}
\section{Reflections}
This course, Protocol Validation, gave us an insight on the modelling of distributed systems, how to write specifications and validate protocols. We learnt how to model in mCRL2 and use the toolset available. 
This course was given in a self study manner this academic year (2021-2022), due to which we struggle a bit. It was advised to take the course next year but due to some constraints we continued with the course and tried to finish it by giving our best shot.
With many ups and downs in doing this assignment, we learnt quite a lot and for that we would like to thank course coordinator, Dr. WJ Fokkink, and teaching assistant Wolf bij 't Vuur. 
We tried our best to finish the mCRL specification but could only model the Console input controller, the sensor input controller and the output controller. We couldn't finish the logic controller as it showed too many errors. We have tried till the last minute to finish it and make it work. We are hence submitting till what we have done. We respect the firm deadline and submitting the report. We will continue working on the assignment and try to do it perfectly. If re-submission is allowed and the assignment is finished before the re-submission deadline, we will submit the more perfect model. Even otherwise, we would work till we have learnt to model this assignment problem. 

\break

\end{document}
